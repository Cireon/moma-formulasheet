\documentclass{article}

\usepackage{amsmath}
\usepackage{amssymb}
\usepackage{amsthm}
\usepackage{enumerate}
\usepackage[a4paper]{geometry}

\newtheorem{theorem}{Theorem}[section]
\newtheorem{lemma}{Lemma}[section]
\newtheorem{corollary}{Corollary}[section]

\theoremstyle{definition}
\newtheorem{definition}{Definition}[section]
\newtheorem{procedure}{Procedure}[section]

\DeclareMathOperator{\vers}{vers}
\DeclareMathOperator{\tr}{tr}
\DeclareMathOperator{\Tran}{Tran}
\DeclareMathOperator{\Rot}{Rot}
\DeclareMathOperator{\Screw}{Screw}

\title{Motion and Manipulation Formula Sheet}
\author{Tom Rijnbeek}

\begin{document}

\section{Trigonometry}

\subsection{Common values}
\begin{tabular}{r|ccccc}
$\phi$ & $0$ & $\frac{\pi}{6}$ & $\frac{\pi}{4}$ & $\frac{\pi}{3}$ & $\frac{\pi}{2}$ \\ \hline
$\sin \phi$ & $0$ & $\frac{1}{2}$ & $\frac{\sqrt{2}}{2}$ & $\frac{\sqrt{3}}{2}$ & $1$ \\
$\cos \phi$ & $1$ & $\frac{\sqrt{3}}{2}$ & $\frac{\sqrt{2}}{2}$ & $\frac{1}{2}$ & $0$
\end{tabular}

\subsection{Sum formulas}
\begin{align*}
	\sin(\alpha + \beta) &= \sin(\alpha)\cos(\beta) + \cos(\alpha)\sin(\beta) \\
	\cos(\alpha + \beta) &= \cos(\alpha)\cos(\beta) - \sin(\alpha)\sin(\beta) \\
	\cos(\alpha - \beta) &= \cos(\alpha)\cos(\beta) + \sin(\alpha)\sin(\beta) \\
	\sin(\alpha - \beta) &= \sin(\alpha)\cos(\beta) - \cos(\alpha)\sin(\beta)
\end{align*}

\section{Kinematics}

\subsection{Notations}
Fixed (or world, global) frame $F$: $\{f_1, f_2, f_3\}$\newline
Moving (or body, local) frame $M$: $\{m_1, m_2, m_3\}$\newline
We denote the moving coordinates by $\vec{p}$ and the fixed coordinates by $\vec{q}$.

\subsection{Rotations}
\begin{definition}
$R_i(\theta)$ is the rotation matrix about the fixed axis $f_i$ with angle $\theta$:\newline
$R_1(\theta) = 
\begin{pmatrix}
1 & 0 & 0 \\
0 & \cos \theta & -\sin \theta \\
0 & \sin \theta & \cos \theta
\end{pmatrix}$; 
$R_2(\theta) = 
\begin{pmatrix}
\cos \theta & 0 & \sin \theta \\
0 & 1 & 0 \\
-\sin \theta & 0 & \cos \theta
\end{pmatrix}$;\newline
$R_3(\theta) = 
\begin{pmatrix}
\cos \theta & -\sin \theta & 0 \\
\sin \theta & \cos \theta & 0 \\
0 & 0 & 1
\end{pmatrix}$.
\end{definition}

\begin{definition}
$A_i(\theta)$ is the rotation matrix about the moving axis $m_i$ with angle $\theta$:\newline
$A_1(\theta) = 
\begin{pmatrix}
1 & 0 & 0 \\
0 & \cos \theta & \sin \theta \\
0 & -\sin \theta & \cos \theta
\end{pmatrix}$; 
$A_2(\theta) = 
\begin{pmatrix}
\cos \theta & 0 & -\sin \theta \\
0 & 1 & 0 \\
\sin \theta & 0 & \cos \theta
\end{pmatrix}$;\newline
$A_3(\theta) = 
\begin{pmatrix}
\cos \theta & \sin \theta & 0 \\
-\sin \theta & \cos \theta & 0 \\
0 & 0 & 1
\end{pmatrix}$.
\end{definition}

\begin{equation*}
\vec{q} = R_i \vec{p} = A_i^{-1}\vec{p} = A_i^T \vec{p}
\end{equation*}
\begin{equation*}
\vec{p} = A_i \vec{q} = R_i^{-1}\vec{q} = R_i^T \vec{q}
\end{equation*}

\begin{theorem}\label{thm_succesive-rotation}\hfill
\begin{enumerate}[{(a)}]
\item Given local coordinates $\vec{p}$. Then the global coordinates of these coordinates after $n$ rotations $R_1, ..., R_n$ about the global axes can be found as $\vec{q} = R_n \cdots R_2 R_1 \vec{p}$.\newline
\item Given global coordinates $\vec{q}$. Then the local coordinates of these coordinates after $n$ rotations $A_1, ..., A_n$ about the local axes can be found as $\vec{p} = A_n \cdots A_2 A_1 \vec{q}$.\newline
\end{enumerate}
\end{theorem}

\begin{corollary}
From $\vec{p} = A_n \cdots A_2 A_1 \vec{q}$ (theorem \ref{thm_succesive-rotation}a) follows \newline
\begin{equation*}
\begin{array}{lcrl}
\vec{q} & = & \left( A_n \cdots A_2 A_1 \right)^{-1} & \vec{p} \\
& = & A_1^{-1} A_2^{-1} \cdots A_n^{-1} & \vec{p} \\
& = & R_1 R_2 \cdots R_n & \vec{p}.
\end{array}
\end{equation*}
\end{corollary}

\subsection{Orientations}
\begin{theorem}
The rotation matrix $R$ that maps moving coordinates to fixed coordinates with an angle $\phi$ about axis $\bar{u}$ (unit vector in global space) is given by:\newline
\begin{equation*}
R = I \cos \phi + \bar{u}\bar{u}^T \vers \phi + \tilde{u} \sin \phi
\end{equation*}
where
\begin{equation*}
\vers \phi = 1 - \cos \phi = 2 \sin^2 \frac{\phi}{2}
\end{equation*}
\begin{equation*}
\tilde{u} = -\tilde{u}^T = \begin{pmatrix}
0 & -u_3 & u_2 \\
u_3 & 0 & -u_1 \\
-u_2 & u_1 & 0
\end{pmatrix}
\end{equation*}
\end{theorem}

\begin{theorem}
Given the rotation matrix $R$, the axis $\bar{u}$ and angle $\phi$ can be retrieved using:
\begin{equation*}
\cos \phi = \frac{1}{2}(\tr(R) - 1)
\end{equation*}
\begin{equation*}
\tilde{u} = \frac{1}{2 \sin \phi}(R - R^T)
\end{equation*}
\end{theorem}

\begin{definition}
Given rotation angle $\phi$ and axis $\bar{u}$, the \emph{Euler parameters} are given by:
\begin{equation*}
\begin{array}{lclr}
e_0 & = & \cos \frac{\phi}{2} & \\
e_i & = & u_i \sin \frac{\phi}{2} & \mbox{ for } i = 1, 2, 3
\end{array}
\end{equation*}
\end{definition}

\begin{theorem}
If the transformation $R$ for the rotation is given, we can recover the Euler parameters by:
\begin{equation*}
e_0^2 = \frac{1}{4}(\tr(R) + 1)
\end{equation*}
\begin{equation*}
\tilde{e} = \frac{1}{4e_0}(R - R^T)
\end{equation*}
\end{theorem}

\subsection{Rigid transformations}
\begin{lemma}
Assume $F$ to be the default orthonormal basis and $M$ given by $m_1, m_2, m_3$ another orthornormal basis. Then the matrix $R$ that transforms global coordinates into local is given by $R = \begin{pmatrix} m_1 & m_2 & m_3 \end{pmatrix}$.\newline
In general, if $F$ is an orthonormal basis given by $f_1, f_2, f_3$, then
\begin{equation*}
R = \begin{pmatrix}
f_1m_1 & f_1m_2 & f_1m_3 \\
f_2m_1 & f_2m_2 & f_2m_3 \\
f_3m_1 & f_3m_2 & f_3m_3
\end{pmatrix}
\end{equation*}
\end{lemma}

\begin{definition}
A \emph{placement} is a rotation followed by a translation. We represent a placement $\vec{q} = (q_1, q_2, q_3)^T \in \mathbb{R}^3$ by $\vec{q'} = (q_1, q_2, q_3, 1) \in \mathbb{R}^4$.
\end{definition}

\begin{definition}
Translation along a threedimensional $\vec{t} \in \mathbb{R}^3$:
\begin{equation*}
\Tran(t) = \begin{pmatrix}
I_3 & t \\
0_3^T & 1
\end{pmatrix}
\end{equation*}
\begin{equation*}
\Tran(t)^{-1} = \Tran(-t)
\end{equation*}
\end{definition}

\begin{definition}
Rotation by angle $\theta$ about axis $k$:
\begin{equation*}
\Rot_k(\theta) = \begin{pmatrix}
R_k(\theta) & 0_3 \\
0_3^T & 1
\end{pmatrix}
\end{equation*}
\begin{equation*}
\Rot_k^{-1}(\theta) = \Rot_k(-\theta) = \Rot_k^T(\theta)
\end{equation*}
\end{definition}

\begin{procedure}
Initialize $M = F \rightarrow T := I$.\newline
Process next transformation of $M$:\newline
\begin{tabular}{rlcl}
- & rotation by $\theta$ about axis $f_k$ & $\rightarrow$ & $T := \Rot_k(\theta) \cdot T$ \\
- & translation by $c$ units along axis $f_k$ & $\rightarrow$ & $T := \Tran(c \vec{e_k}) \cdot T$ \\
- & rotation by $\theta$ about axis $m_k$ & $\rightarrow$ & $T := T \cdot \Rot_k(\theta)$ \\
- & translation by $c$ units along axis $m_k$ & $\rightarrow$ & $T := T \cdot \Tran(c \vec{e_k})$ \\
\end{tabular}
\end{procedure}

\subsection{Denavit-Hartenberg representation}
\begin{definition}
Joint $k$ connects link $k-1$ and link $k$.
\begin{itemize}
\item Joint angle $\theta_k$: angle of rotation about $z_{k-1}$ to make $x_{k-1}$ parallel with $x_k$.
\item Joint distance $d_k$: distance of translation along $z_{k-1}$ to make $x_{k-1}$ intersect with $x_k$.
\end{itemize}
Link $k$ connects joint $k$ with joint $k+1$.
\begin{itemize}
\item Twist angle $\alpha_k$: angle of rotation about $x_k$ to make $z_{k-1}$ parallel with $z_k$.
\item Link length $a_k$: distance of translation along $x_k$ to make $z_{k-1}$ interesect with $z_k$.
\end{itemize}
\end{definition}

\begin{procedure}
\emph{The Denavit-Hartenberg algorithm} assigns coordinate fromes to links/joints and determines parameters in a systematic way to get a ``standard form'' for all coordinate transforms. Below is an outline of this algoritm; for a more in-depth description please refer to the slides.
\begin{enumerate}
\item joint numbering;
\item assignment of base frame $L_0$;
\item assignment of frames 1 through $k-1$ ($L_1, ..., L_{k-1}$);
\item assignment of tool frame $L_k$;
\item computation of joint and link parameters.
\end{enumerate}
\end{procedure}

\begin{definition}
$\Screw_i(\theta, d) = \Rot_i(\theta) \cdot \Tran(d e_i)$
\end{definition}

\begin{theorem}
Every transformation $T $ from $L_k$ to $L_{k-1}$ can be expressed as two screw operations. I.e. $\Screw_3(\theta_k, d_k) \cdot \Screw_1(\alpha_k, a_k)$ maps coordinates with respect to $L_k$ to coordinates with respect to $L_{k-1}$.
\end{theorem}

\section{Path generation}
\paragraph{Cubic paths}
Given four conditions $q(t_0) = q_0, q(t_f) = q_f, q'(t_0) = v_0, q'(t_f) = v_f$. Fit cubic polynomial \[ q(t) = a_0 + a_1t + a_2t^2 + a_3t^3 \] and solve for $a_0, a_1, a_2, a_3$.

\paragraph{Quadratic paths}
Given three conditions $q(t_0) = q_0, q(t_i) = q_i, q(t_f) = q_f$. Fit quadratic polynomial \[ q(t) = a_0 + a_1t + a_2t^2 \] and solve for $a_0, a_1, a_2$.

\paragraph{General polynomial paths}
Given $n+1$ conditions, fir $n$-degree polynomial and solve for $a_0, a_1, ..., a_n$.

\section{Configutation space}
Let $W$ be the workspace, $A$ a robot with fixed shape, $B$ the collection of obstacles, $C$ the configuration space (space of all possible placements of the robot). Denote a configuration (unique characterization of robot placement by a (minimum) number of parameters) by $q$.

\begin{definition}
A configuration $q$ is \emph{free} if $\forall B_i \in B : A(q) \cap B_i = \emptyset$; a configuration $q$ is \emph{forbidden} if $\exists B_i \in B : A(q) \cap B_i \neq \emptyset$.
\end{definition}

\begin{definition}
Let $CB_i = \left\{ q \in C | A(q) \cap B_i \neq \emptyset \right\}$ be the configuration space obstacle for $B_i$. Then the forbidden space $CB := \bigcup \limits_i CB_i$ and the free space $C_{free} := C \backslash CB$.
\end{definition}

\begin{definition}
The Minkowsky sum of set $P$ and $Q$ is defined as \[ P \oplus Q := \left\{ p + q | p \in P, q \in Q \right\}. \] For convex objects with $m$ and $n$ vertices, this can be calculated in $O(m + n)$ time. If one polygon is non-convex, this becomes $\Omega(mn)$ and for two non-convex polygons the complexity is $\Omega(m^2n^2)$.
\end{definition}

\begin{corollary}
\[ CB_i = B_i \oplus (-A) \]
\end{corollary}

\section{Collision detection}

\begin{lemma}
Objects $A$ and $B$ intersect if and only if $O \in B \oplus (-A)$ and equivalently $O \in A \oplus (-B)$.\newline
If $A$ and $B$ intersect then the point $v$ closest to $O$ on the boundary of $B \oplus (-A)$ corresponds to the shortest translation taking apart $A$ and $B$ ($|v|$ is called the \emph{penetration depth}).\newline
If $A$ and $B$ don't intersect, this point $v$ is the shortest translation to make them intersect. $|v|$ is referred to as the \emph{distance}.
\end{lemma}

\begin{theorem}[Seperating Axis Theorem]
A separating axis of two disjoint objects $A$ and $B$ is a vector $\vec{v}$ such that $A$ and $B$ projected on $\vec{v}$ do not intersect. For any pair of non-intersecting polytopes, there exists a separating axis that is orthogonal to a facet of either polytope, or orthogonal to an edge of each polytope.
\end{theorem}

\begin{definition}
A separating plane leaves objects on two opposite sides. A strong separating plane is non-touching, but is hard to find. Weak separating planes can touch the objects.
\end{definition}

\begin{definition}
Support mapping gives (an) extreme point $S_C(d)$ in $C$ in any direction $d$ (not always uniquely defined).
\end{definition}

\begin{definition}
An $n$-simplex is an $n$-dimensional polytope that can be defined with the least amount of vertices.
\end{definition}

\begin{procedure}[GJK algorithm] The GJK algorithm calculates the penetration depth.
\begin{enumerate}
\item initiale $d$-simplex set $Q$ with up to $d+1$ points from $C$ (in $d$ dimensions);
\item compute point $P$ of minimum norm in the convex hull $CH(Q)$;
\item if $P$ is the origin, exit, return $0$;
\item reduce $Q$ to the smallest subset $Q'$ of $Q$ such that $P \in CH(Q')$;
\item let $V = S_C(-P)$ be a supporting point in direction $-P$;
\item if $V$ is not more extreme in direction $-P$ than $P$ itself, exit, return $||P||$;
\item add $V$ to $Q$, go to step 2.
\end{enumerate}
\end{procedure}

\section{Force closure}
\begin{definition}
A screw is a line in space with an associated pitch, which is the ratio of linear to angular quantities. A twist is a screw plus a scalar magnitude, giving a rotation about the screw axis plus a translation along the screw axis.
\end{definition}

\begin{definition}
Given a line $l$: $\vec{x} = \vec{p} + \lambda \vec{q}$. The Pl\"ucker coordinates of $l$ are $(\vec{q}, \vec{q_0})$, where $\vec{q_0} = \vec{p} \times \vec{q}$. The magnitude $\frac{|\vec{q_0}|}{|\vec{q}|}$ gives the distance from the origin to the line $l$. $\vec{q_0} = \vec{0}$ corresponds to a line through the origin; $\vec{q} = \vec{0}$ corresponds to a line at infinity.
\end{definition}

\begin{definition}
The screw coordinates are defined as $(\vec{s}, \vec{s_0})$, where $\vec{s} = \vec{q}$ and $\vec{s_0} = \vec{q_0} + p\vec{q}$ (where $p$ is the pitch).\newline
Given the screw coordinates, we can retrieve the pitch by $p = \frac{\vec{s} \cdot \vec{s_0}}{\vec{s} \cdot \vec{s}}$.
\end{definition}

\begin{lemma}
$\vec{f_1}, \vec{f_2}, \vec{f_3}$ can resist all external forces if they form a positive basis:
\begin{itemize}
\item any vector $\vec{f}$ can be written as $-\vec{f} = k_1\vec{f_1} + k_2\vec{f_2} + k_3\vec{f_3}$ for some $k_1, k_2, k_3 \leq 0$, or equivalently
\item $k_1\vec{f_1} + k_2\vec{f_2} + k_3\vec{f_3} = O$ for some $k_1, k_2, k_3 > 0$, or equivalently
\item the convex hull of $\vec{f_1}, \vec{f_2}, \vec{f_3}$ has $O$ in its interior.
\end{itemize}
\end{lemma}

\begin{definition}
A force $f$ applied on a part in point $p$ are expressed by a wrench, consisting of a force $f$ and moment $p \times f$: \[ w = \begin{pmatrix}f\\p \times f\end{pmatrix} \]
\end{definition}

\begin{definition}
Force closure: wrenches $w_1, ..., w_k$ should be able to resist any external force, so
\begin{itemize}
\item $w_1, ..., w_k$ form a positive basis for wrench space, so
\item the convex hull of $w_1, ..., w_k$ has $O$ in its interior.
\end{itemize}
\end{definition}
\end{document}